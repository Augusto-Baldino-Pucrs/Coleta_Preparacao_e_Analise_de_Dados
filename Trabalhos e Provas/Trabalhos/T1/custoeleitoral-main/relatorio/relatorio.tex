\documentclass[12pt]{article}
\usepackage[a4paper,top=2.5cm,bottom=3.5cm,left=2.5cm,right=2.5cm]{geometry}
\usepackage{titlesec}
\usepackage{enumitem}
\usepackage{graphicx}
\usepackage{lipsum}
\usepackage{xcolor}
\usepackage{fancyhdr}
\usepackage{booktabs}
\usepackage{longtable}
\usepackage{array}
\usepackage{caption}
\usepackage[hidelinks]{hyperref}

\renewcommand{\contentsname}{Sumário}

\pagestyle{fancy}
\setlength{\footskip}{35pt}
\fancyhf{}
\fancyfoot[C]{%
  {\scriptsize\color{gray}
    Av. Ipiranga, 6681\thinspace|\thinspace Prédio 30\thinspace|\thinspace Bloco C\thinspace|\thinspace Sala 101\\
    CEP 90619-900\thinspace|\thinspace Porto Alegre, RS\thinspace|\thinspace Brasil\\
    Fone: (51) 3320-3558\thinspace|\thinspace E-mail: politecnica@pucrs.br\thinspace|\thinspace www.pucrs.br/politecnica%
  }
}

\begin{document}

\begin{titlepage}
    \thispagestyle{fancy}

    \centering
    {\LARGE\textbf{PROJETO EM CIÊNCIA DE DADOS}}\\[1cm]
    
    \textbf{SEMESTRE:} 2025/2\\[0.5cm]
    \textbf{PROJETO:} \texttt{O custo das eleições}\\[0.5cm]
    \textbf{COMPONENTES DO GRUPO:}\\
    Augusto Peroni Baldino\\
    Bruno Grigoletti Laitano\\
    Daniel Chin Tay Lee\\
    João Pedro Zarth Dias\\[1cm]

    \begin{flushleft}
    \textbf{Breve descrição do problema:} Este trabalho extensionista consiste em realizar processos de coleta, preparação e análise de dados a partir dos registros abertos da Câmara dos Deputados, com informações a respeito do trabalho de deputados federais. De forma específica, o texto analisa o impacto das quatro últimas eleições gerais (2010, 2014, 2018 e 2022) sobre as atividades parlamentares, em especial sobre o valor líquido gasto por cada representante.\\[0.5cm]

    \textbf{Breve descrição da solução proposta:} Realizamos o \emph{download} de todos os arquivos existentes no portal em formato \texttt{.json} referentes aos gastos parlamentares dos últimos dezessete anos \textemdash \ de 2008 a 2025. Esse período de tempo compreende quatro eleições gerais. Analisamos a variância dos gastos realizados em anos eleitorais e em anos não eleitorais, buscando compreender as razões da divergência de valores, em especial o impacto do pleito sobre as atividades de deputados federais.\\[0.5cm]

    \textbf{Fases da Metodologia CRISP-DM:}
        \renewcommand{\arraystretch}{1.5}
        \begin{longtable}{>{\raggedright\arraybackslash}p{0.25\linewidth} >{\raggedright\arraybackslash}p{0.6\linewidth} >{\centering\arraybackslash}p{0.1\linewidth}}
            \toprule
            \textbf{Fase} & \textbf{Tarefas} & \textbf{\%} \\
            \midrule
            \endfirsthead
            
            \toprule
            \textbf{Fase} & \textbf{Tarefas} & \textbf{\% Concluído} \\
            \midrule
            \endhead
            
            Compreensão do Negócio & Entendimento dos objetivos e requisitos sob a perspectiva do negócio & 100\% \\
            Compreensão dos Dados & Avaliação das fontes de dados disponíveis e determinação de coletas adicionais & 0\% \\
            Preparação dos Dados & Limpeza dos dados e preparo do \emph{dataset} a ser utilizado na fase de modelagem & 0\% \\
            \bottomrule
        \end{longtable}


    \textbf{Resumo do que foi concluído até o momento:} Recuperar as entregas pretendidas e comentar o que foi concluído até o momento, destacando os principais desafios, as dificuldades enfrentadas e como foram/serão superadas, e o status de cada item conforme o seu planejamento.
    \end{flushleft}
\end{titlepage}

\tableofcontents
\newpage

\newpage
\section{Compreensão do Negócio}

\subsection{Background}
Visão geral sobre o contexto do projeto. Qual a área do projeto? Quais problemas foram identificados? Por que mineração parece ser uma boa solução?

\subsection{Objetivos de negócio e critérios de sucesso}
Descrição dos objetivos do negócio. Para cada objetivo identificado, definir critérios de sucesso.

\subsection{Inventário de recursos}
Listar pessoas, fontes de dados, equipamentos e demais recursos disponíveis.

\subsection{Requisitos, suposições e restrições}
Descrever requisitos gerais, suposições feitas e restrições existentes.

\subsection{Terminologia}
Apresentar os termos de domínio utilizados no projeto.

\subsection{Objetivos de mineração e critérios de sucesso}
Indicar os objetivos em termos de mineração de dados e seus respectivos critérios de sucesso.

\subsection{Plano de Projeto}
Descrever etapas previstas, duração, recursos necessários, entradas/saídas e dependências.

\subsection{Avaliação inicial de técnicas e ferramentas}
Quais ferramentas e técnicas o grupo pretende usar e como pretende usá-las.

\subsection{Autoavaliação}
\begin{itemize}
    \item Augusto Peroni Baldino | Nota: \underline{\hspace{1cm}} / 10,0.
    \item Bruno Grigoletti Laitano | Nota: \underline{\hspace{1cm}} / 10,0.
    \item Daniel Chin Tay Lee | Nota: \underline{\hspace{1cm}} / 10,0.
    \item João Pedro Zarth Dias | Nota: \underline{\hspace{1cm}} / 10,0.
\end{itemize}

\subsection*{Status do escopo}
Indicar se o grupo acredita que cumprirá 100\% do escopo pretendido e justificar a resposta.

\end{document}